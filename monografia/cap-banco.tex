%% ------------------------------------------------------------------------- %\%
\chapter{Banco de Dados}
\label{cap:baco_de_dados}

%% ------------------------------------------------------------------------- %%
\section{Modelo Conceitual}
\label{sec:modelo_conceitual}

TODO Aqui vai a figura do diagrama conceitual do projeto (DER) que demonstra as rela��es entre as entidades. 

% %% ------------------------------------------------------------------------- %%
\section{Modelo Relacional}
\label{sec:modelo_relacional}

TODO Aqui eu fa�o o modelo relacional (figura) demonstrando as tabelas do banco de dados e suas respectivas
chaves prim�rias e/ou estrangeiras. 

% %% ------------------------------------------------------------------------- %%
\section{Rela��es}
\label{sec:relacoes}

	Foram necess�rias sete tabelas para armazenar os dados necess�rios para a implementa��o do jogo.

\subsection{Jogador}
	
\begin{table}[!h]
\large
\centering
\begin{tabular}{|c|c|c|c|c|}

\hline
\underline{id} & nome & moedas & epeciais & foto \\
\hline

\end{tabular}
\caption{Tabela Jogador}
\end{table}

\begin{itemize}
\item \textbf{id} - 
\item \textbf{nome} - 
\item \textbf{moedas} - 
\item \textbf{epeciais} - 
\item \textbf{foto} - 
\end{itemize}

\subsection{TipoCarta}

\begin{table}[!h]
\large
\centering
\begin{tabular}{|c|c|}

\hline
\underline{id} & tipo\\
\hline

\end{tabular}
\caption{Tabela TipoCarta}
\end{table}

\begin{itemize}
\item \textbf{id} - 
\item \textbf{tipo} - 
\end{itemize}

\subsection{Carta}

\begin{table}[!h]
\large
\centering
\begin{tabular}{|c|c|c|c|}

\hline
\underline{id} & nome & id\_tipo\_carta & link\_foto\\
\hline

\end{tabular}
\caption{Tabela Carta}
\end{table}

\begin{itemize}
\item \textbf{id} - 
\item \textbf{nome} - 
\item \textbf{id\_tipo\_carta} - 
\item \textbf{link\_foto} - 
\end{itemize}

\subsection{Dicas}

\begin{table}[!h]
\large
\centering
\begin{tabular}{|c|c|c|}

\hline
\underline{id} & id\_carta & texto\\
\hline

\end{tabular}
\caption{Tabela Dicas}
\end{table}

\begin{itemize}
\item \textbf{id} - 
\item \textbf{id\_carta} - 
\item \textbf{texto} - 
\end{itemize}

\subsection{Desafios}

\begin{table}[!h]
\large
\centering
\begin{tabular}{|c|c|c|c|
c|c|}

\hline
\underline{id\_jogador1} & \underline{id\_jogador2} & id\_carta & pontuacao1 \\ 
\hline
pontuacao2 & status \\
\cline{1-2}

\end{tabular}
\caption{Tabela Desafios}
\end{table}

\begin{itemize}
\item \textbf{id\_jogador1} - 
\item \textbf{id\_jogador2} - 
\item \textbf{id\_carta} - 
\item \textbf{pontuacao1} - 
\item \textbf{pontuacao2} - 
\item \textbf{status} - 
\end{itemize}

\subsection{Historico\_Jogo}

\begin{table}[!h]
\large
\centering
\begin{tabular}{|c|c|c|c|}

\hline
\underline{id\_jogador1} & \underline{id\_jogador2} & vitorias1 & vitorias2 \\
\hline

\end{tabular}
\caption{Tabela Historico\_Jogo}
\end{table}

\begin{itemize}
\item \textbf{id\_jogador1} - 
\item \textbf{id\_jogador2} - 
\item \textbf{vitorias1} - 
\item \textbf{vitorias2} - 
\end{itemize}

\subsection{Historico\_Estatistica}

\begin{table}[!h]
\large
\centering
\begin{tabular}{|c|c|c|c|c|c|}

\hline
\underline{id\_jogador} & \underline{id\_tipo\_carta} & jogadas & acertos \\
\hline

\end{tabular}
\caption{Tabela Historico\_Estatistica}
\end{table}

\begin{itemize}
\item \textbf{id\_jogador} - 
\item \textbf{id\_tipo\_carta} - 
\item \textbf{jogadas} - 
\item \textbf{acertos} - 
\end{itemize}


% %% ------------------------------------------------------------------------- %%
\section{Arquivos de cria��o e de inser��o}
\label{sec:arquivos}

	Foram criados dois arquivos com extens�o sql para fazer a cria��o das tabelas do 
banco de dados: 
\begin{itemize}
\item{\textbf{ModeloFisico}} - Arquivo de cria��o do banco 
\item{\textbf{Populate}} - Arquivo que popula o banco de dados com alguns jogadores e cartas para
que os testes ... 
\end{itemize} 


% %% ------------------------------------------------------------------------- %%