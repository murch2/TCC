%% ------------------------------------------------------------------------- %\%
\chapter{Subjtivo}
\label{cap:Subjtivo}

%% ------------------------------------------------------------------------- %%
\section{Desafios}
\label{sec:desafios}

	Os principais desafios encontrados durante o desenvolvimento desse projeto foram: 

\subsection{N�o conhecimento das linguagens utilizadas}
	Uma das principais dificuldades no andamento do projeto foi justamente um de seus maiores 
objetivos: O aprendizado de linguagens que tive pouco contato durante a gradua��o. As duas
principais linguagens utilizadas no projeto foram Java e PHP. Durante o desenvolvimento de 
um jogo relativamente extenso, o desconhecimento de estruturas e facilidades 
oferecidas pela linguagem s�o empecilhos que tornam tarefas simples demorarem mais tempo
do que o esperado.  

\subsection{Documenta��o da engine}
	A andEngine [Ref ao site da andEngine] possui grande n�mero de usu�rios, e um f�rum bastante 
ativo, o que ajuda muito em momentos de d�vidas sobre como instanciar certos objetos e utilizar
alguns de seus m�todos. Por�m a falta de documenta��o faz com que o processo de implementa��o seja 
muito ligado a pesquisa sobre a engine. Este foi um ponto que fez com que atividades, que
a princ�pio, supondo que existisse uma documenta��o da biblioteca, seriam r�pidas, 
se tornassem dependentes de repetidas buscas e experimentos.

\subsection{Interface gr�fica}
	A total falta de conhecimento sobre ferramentas de modelagem gr�fica foi outro ponto
impactante no trabalho. Como a parte gr�fica de um jogo � um fator determinante 
para seu sucesso, empreguei uma parte do tempo reservado ao trabalho aprendendo a
utilizar o programa GIMP.
