%% ------------------------------------------------------------------------- %%
\chapter{Tecnologias Utilizadas}
\label{cap:tecnologias_utilizadas}

%% ------------------------------------------------------------------------- %%
\section{Ferramentas}
\label{sec:ferramentas}

\subsection{ADT}
\label{sec:adt}
	O ambiente de desenvolvimento do Android SDK, disponibiliza as ferramentas
necess�rias para come�ar a desenvolver as aplica��es na plataforma utilizando a linguagem de 
programa��o Java. 

	O SDK inclui funcionalidades ut�is na cria��o de aplicativos como por exemplo emuladores 
de aparelhos, ferramentas de depura��o, visualiza��o da utiliza��o de mem�ria, an�lise de desempenho, 
dentre outras. 

	O ADT \footnote{ADT: \url{http://developer.android.com/tools/sdk/eclipse-adt.html}} � 
um plugin para a IDE Eclipse\footnote{Eclipse: \url{https://www.eclipse.org/}} disponibilizado pela Google para o desenvolvimento de aplicativos android.


\subsubsection{Eclipse}
\label{subsec:eclipse}
	Falar do Eclipse aqui.


%% ------------------------------------------------------------------------- %%
\subsection{AndEngine}
\label{sec:andEngine}

TODO falar sobre a andEngine, o que �, quem escreveu, que � open source, fazer o link de download e 
explicar um pouco da sua arquitetura (Talvez para explicar sobre a arquitetura seja necess�rio dividir em subt�picos)

%% ------------------------------------------------------------------------- %%
\subsection{PostgreSQL}
\label{sec:postgreSQL}

TODO falar sobre o PostgreeSQL, � open source, criado no ano x, utilizei a vers�o y e � mantido so site z. 

%% ------------------------------------------------------------------------- %%
\subsection{Nginx}
\label{sec:nginx}

TODO falar sobre o Nginx. 

%% ------------------------------------------------------------------------- %%
\subsection{Facebook SDK}
\label{sec:facebook_sdk}

TODO falar sobre o Facebook SDK.

%% ------------------------------------------------------------------------- %%
% \subsection{GIMP}
% \label{sec:gimp}

% TODO falar sobre o GIMP (Talvez)

%% ------------------------------------------------------------------------- %%
\section{Linguagens de Programa��o}
\label{sec:linguagens_de_programacao}

\subsection{Java}
\label{java}

TODO Falar sobre a linguagem de programa��o orientada a objetos Java. 

%% ------------------------------------------------------------------------- %%
\subsection{PHP}
\label{php}

TODO Falar sobre a linguagem de programa��o PHP. 

